\section{Ph.D.}



\subsection{}
%%%%%%%%%%%%%%%%%%%%%%%%%%%%%%%%%%%%%%%%%%%%%%%%%%%%%%%%
{
\paper{
Xochicale 2018 in {\bf Preprint PhD thesis (Zenodo)}
}
\begin{frame}{Thesis Outline}
    \begin{figure}
        \includegraphicscopyright[width=0.9\linewidth]{phd/thesis-structure/thesis-structure}{}
	%\caption{Thesis outline.} 
   \end{figure}
\end{frame}
}




\subsection{}
%%%%%%%%%%%%%%%%%%%%%%%%%%%%%%%%%%%%%%%%%%%%%%%%%%%%%%%%
{
\paper{
Davids et al. 2003 in {\bf Sport Medicine}; 
Newell et al. 2001 in {\bf Hum Mov Sci}
%Xochicale 2018 in {\bf Preprint PhD thesis (Zenodo)}
}
\begin{frame}{Modelling Movement Variability}
    \begin{figure}
        \includegraphicscopyright[width=0.9\linewidth]{phd/davids2002/fig1}{}
	\caption{Newell's model of variability of physical performances} 
   \end{figure}
\end{frame}
}




\subsection{}
%%%%%%%%%%%%%%%%%%%%%%%%%%%%%%%%%%%%%%%%%%%%%%%%%%%%%%%%
{
\paper{
Stergiou et al. 2006 in {\bf Neurologic Physical Therapy} 
Stergiou and Decker 2011 in {\bf Human Movement Science}}
\begin{frame}{Modelling Movement Variability}
    \begin{figure}
        \includegraphicscopyright[width=0.9\linewidth]{phd/stergiou2006/drawing}{}
	\caption{Theoretical Model of Optimal Movement Variability} 
   \end{figure}
\end{frame}
}





\subsection{}
%%%%%%%%%%%%%%%%%%%%%%%%%%%%%%%%%%%%%%%%%%%%%%%%%%%%%%%%
{
{
\paper{
Xochicale 2018 in {\bf Preprint PhD thesis (Zenodo)}
}
\begin{frame}{Nonlinear Analysis}

%Entropy measures to quantify regularity and complexity of time series.
%"Is there a best tool to measure variability?" (Caballero et al. 2014, p. 67)

There is no best tool to measure MV and unification of tools is still
an open question (Caballero et al. 2014; Wijnants et al. 2009)
which led me (i) to explore different nonlinear analyses 
to measure MV and (ii) to understand their strengths and weaknesses. 

\begin{itemize}
	\item Approximate Entropy (Pincus 1991, 1995)
	\item Sample Entropy (Richman and Moorman, 2000)
	\item Multiscale Entropy (Costa et al., 2002)
	\item Detrended Fluctation Analysis (Peng et al., 1995)
	\item Largest Lyapunov exponent (Stergiou, 2016)
	\item Recurrence Quantification Analysis (Zbilut and Webber et al., 1992)
	%\item {\bf Recurrence Domains of Dynamical Systems by Symbolic Dynamics} 
	%	(beim Graben and Hutt 2013)

\end{itemize}


\end{frame}
}



\subsection{}
%%%%%%%%%%%%%%%%%%%%%%%%%%%%%%%%%%%%%%%%%%%%%%%%%%%%%%%%
{
\paper{Takens 1981 in {\bf Dynamical Systems and Turbulence}; 
Casdagli 1991 in {\bf Physica D};
Frank et al. 2010 in {\bf AAAI Conference on Artificial Intelligence};
Sama et al. 2013 in {\bf Neurocomputing};
Cao 1997 in {\bf Physica D};
Kabiraj et al. 2012 in {\bf Chaos};
Eckmann et al. 1987 in {\bf Europhysics Letters}
}

\begin{frame}{Nonlinear Analysis}
    \begin{figure}
        \includegraphicscopyright[width=1.1\linewidth]{phd/nonlinear-analyses/main-method}
	{Figure is adapted from Xochicale 2018 in Preprint PhD thesis (Zenodo).} 
	%\caption{Figure adapted from (Casdagli et al. 1991; Quintana-Duque (2012); Uzal et al. 2011)} 
   \end{figure}
	
\end{frame}
}




\subsection{}
%%%%%%%%%%%%%%%%%%%%%%%%%%%%%%%%%%%%%%%%%%%%%%%%%%%%%%%%
{
\paper{Xochicale 2018 in {\bf Preprint Ph.D dissertation (Zenodo)} }

\begin{frame}{Human-Humanoid Imitation Activities}

\small
23 right-handed healthy participants were invited to imitate 
simple arm horizontal and vertical movements from an humanoid.
    \begin{figure}
        \includegraphicscopyright[width=0.9\linewidth]{phd/experiment/hri}{}
	\caption[PA]{(A/C) Front-to-Front HHI  
		for Horizontal/Vertical Movements.
		(B/D) Humanoid robot performing Horizontal/Vertical arm movements
		}
   \end{figure}
	
\end{frame}
}



%\section{Results}

\subsection{}
%%%%%%%%%%%%%%%%%%%%%%%%%%%%%%%%%%%%%%%%%%%%%%%%%%%%%%%%
{
\paper{
Xochicale 2018 in {\bf Preprint PhD thesis (Zenodo)}
}

\begin{frame}{Results: 3D RQA entropies}
    \begin{figure}
        %\centering
        \includegraphicscopyright[width=0.9\linewidth]{phd/results/3dentropies}{}
	%\caption{(A) Normalised, (B) \textt{sgolay(p=5,n=25)}, and (C) \textt{sgolay(p=5,n=159)} } 
   \end{figure}
	
\end{frame}
}



\subsection{}
%%%%%%%%%%%%%%%%%%%%%%%%%%%%%%%%%%%%%%%%%%%%%%%%%%%%%%%%
{
\paper{
Xochicale 2018 in {\bf Preprint PhD thesis (Zenodo)}
}

\begin{frame}{FIRST Open Access Preprint PhD 
		Dissertation at UoB (since 1901)}
    \begin{figure}
        %\centering
        \includegraphicscopyright[width=1.0\linewidth]{phd/oathesis}{}
	%\caption{(A) Normalised, (B) \textt{sgolay(p=5,n=25)}, and (C) \textt{sgolay(p=5,n=159)} } 
   \end{figure}
	
\end{frame}
}





\subsection{}
%%%%%%%%%%%%%%%%%%%%%%%%%%%%%%%%%%%%%%%%%%%%%%%%%%%%%%%%
{
%\paper{}

\begin{frame}{Publications}

\tiny
 
PEER-REVIEW CONFERENCE PAPERS
\begin{itemize}	
	\item Towards the Analysis of Movement Variability in Human-Humanoid Imitation Activities 
	(HAI2017) 
	\item Towards the Quantification of Human-Robot Imitation Using Wearable Inertial Sensors (HRI2017)
	\item Analysis of the Movement Variability in Dance Activities using Wearable Sensors (WeRob2016)
	\item Understanding Movement Variability of Simplistic Gestures Using an Inertial Sensor (PerDis2016)
\end{itemize}

PREPRINTS 
\begin{itemize}	
	\item Strengths and weaknesses of Recurrence Quantification Analysis in the context of human-humanoid interaction
	(ArXiv, October 2018)
\end{itemize}

TALKS 
\begin{itemize}	
	\item Quantifying the Inherent Chaos of Human Movement Variability \\
	15th Experimental Chaos and Complexity Conference 
	\item Towards the Analysis of Movement Variability for Facial Expressions with
	Nonlinear Dynamics \\
	The 7th Consortium of European Research on Emotion Conference 
\end{itemize}


	
\end{frame}
}



% PUBLICATIONS

%Accepted Conferences Publications
%


%\subsection{}
%%%%%%%%%%%%%%%%%%%%%%%%%%%%%%%%%%%%%%%%%%%%%%%%%%%%%%%%%
%{
%\paper{
%Xochicale et al. 2018 in {\bf Poster Conferences at University of Birmingham} 
%}
%\begin{frame}{Quantifying Emotion and Movement Variability in HRI}
%    \begin{figure}
%        \includegraphicscopyright[width=1.0\linewidth]{experiment/emov}{}
%   \end{figure}
%\end{frame}
%}
%
%


%\subsection{}
%%%%%%%%%%%%%%%%%%%%%%%%%%%%%%%%%%%%%%%%%%%%%%%%%%%%%%%%%
%{
%\paper{Takens 1981 in {\bf Dynamical Systems and Turbulence}; 
%Casdagli 1991 in {\bf Physica D};
%Frank et al. 2010 in {\bf AAAI Conference on Artificial Intelligence};
%Sama et al. 2013 in {\bf Neurocomputing};
%Cao 1997 in {\bf Physica D};
%Kabiraj et al. 2012 in {\bf Chaos};
%Eckmann et al. 1987 in {\bf Europhysics Letters}
%}
%
%\begin{frame}{Recurrence Quantification Analysis}
%    \begin{figure}
%        \includegraphicscopyright[width=1.2\linewidth]{nonlinear-analyses/main-method}
%	{Figure is adapted from Xochicale 2018 in Preprint PhD thesis.} 
%	%\caption{Figure adapted from (Casdagli et al. 1991; Quintana-Duque (2012); Uzal et al. 2011)} 
%   \end{figure}
%	
%\end{frame}
%}
%


%
%\subsection{SSRT and UTDE}
%%%%%%%%%%%%%%%%%%%%%%%%%%%%%%%%%%%%%%%%%%%%%%%%%%%%%%%%%
%{
%\paper{Takens F 1981 in {\bf Dynamical Systems and Turbulence}; Casdagli 1991 in {\bf Physica D}}
%
%\begin{frame}{Takens's Theorem}
%
%\LARGE
%%%********************************[EQUATION]************************************
%\begin{equation*}\label{eq:measurement}
%	s(t)= f^{t}[ s(0) ]
%	%x(t)=h[ f^{t}s(0)  ]
%\end{equation*}
%%%********************************[EQUATION]************************************
%%\vspace{0.1mm}
%\normalsize
%\begin{itemize}
%\item $s$ represents a trajectory which evolves in an unknown $d-$dimensional manifold $M$
%\item $f^t$ is a evolution function with time evolution $t$
%\end{itemize}
%Then 
%\LARGE
%%%********************************[EQUATION]************************************
%\begin{equation*}\label{eq:measurement}
%	x(t)=h[s(t)]
%\end{equation*}
%%%********************************[EQUATION]************************************
%%\vspace{0.1mm}
%\normalsize
%\begin{itemize}
%\item $x(t)$ scalar time series in $\mathbb{R}$ 
%\item $h$ is a function defined on the trajectory $s(t)$
%\end{itemize}
%
%
%%where $h$ is a function, $h: M \rightarrow \mathbb{R}$, defined
%%on the evolution function $f^t$ amd
%	
%\end{frame}
%}
%
%
%\subsection{}
%%%%%%%%%%%%%%%%%%%%%%%%%%%%%%%%%%%%%%%%%%%%%%%%%%%%%%%%%
%{
%\paper{Takens F 1981 in {\bf Dynamical Systems and Turbulence}; Casdagli 1991 in {\bf Physica D}}
%
%\begin{frame}{State Space Reconstruction Theorem}
%
%Uniform time-delay embedding matrix
%$\boldsymbol{X}(t) = \{ x(t), x(t-\tau) , ...,x(t - (m-1)\tau  ) \}$ 
%defines a map $\Phi: M \rightarrow \mathbb{R}^m$ such that 
%\begin{eqnarray*}
%\boldsymbol{X}(t) = \Phi(s(t))$
%\end{eqnarray*}
%where $\Phi$ is a diffeomorphic map whenever $\tau > 0$ 
%and $m > 2d_{box}$ and $d_{box}$ is the box-counting dimension of $M$.
%
%\end{frame}
%}
%
%
%
%
%\subsection{}
%%%%%%%%%%%%%%%%%%%%%%%%%%%%%%%%%%%%%%%%%%%%%%%%%%%%%%%%%
%{
%\paper{Frank et al. 2010 in {\bf AAAI Conference on Artificial Intelligence} and Sama et al. 2013 in {\bf Neurocomputing} }
%
%\begin{frame}{Uniform Time-Delay Embedding (UTDE)}
%
%For a given discrete time series $\{x_n\}_{n=1}^{N} = [x_1 , x_2, \dots, x_N]$
%of sample length $N$, a uniform time-delay embedding matrix is defined as 
%\begin{eqnarray*}
% \mathbf{X}^{m}_{\tau}
%  = \begin{pmatrix} \nonumber
%      \tilde{x}_n  & \\
%      \tilde{x}_{n-\tau}  & \\
%      \vdots  &  \\
%      \tilde{x}_{n-(m-1)\tau} & \\
%      \end{pmatrix}^T
%\end{eqnarray*}
%where $m$ is the \textbf{embedding dimension}  and  $\tau$ is the \textbf{ embedding delay}.
%
%
%The sample length for $\tilde{x}(n-i\tau)$, where $0 \leq i \leq (m-1)$, is $N-(m-1)\tau$,
%and the dimensions of $\mathbf{X}^{m}_{\tau}$ are ($m$,$(N-(m-1)\tau)$).
%
%\end{frame}
%}
%
%
%\subsection{Estimation of Embedding Parameters}
%%%%%%%%%%%%%%%%%%%%%%%%%%%%%%%%%%%%%%%%%%%%%%%%%%%%%%%%%
%{
%\paper{Cao L. 1997 in {\bf Physica D}; Kabiraj et al. 2012 in {\bf Chaos}}
%
%\begin{frame}{Estimation of Embedding Parameters }
%
%\begin{block}{False Nearest Neighbours (FNN) for $m$}
%Unfold the attractor (i.e. evolving trajectories
%in a state space).
%\end{block}
%
%\begin{block}{Average Mutual Information (AMI) for $\tau$}
%Maximize the information in the RSSs.
%\end{block}
%	
%\end{frame}
%}
%
%
%
%\subsection{FNN and AMI}
%%%%%%%%%%%%%%%%%%%%%%%%%%%%%%%%%%%%%%%%%%%%%%%%%%%%%%%%%
%{
%%\paper{Xochicale M 2018, {\bf PhD thesis}}
%
%\begin{frame}{False Nearest Neighbours (FNN) for embedding dimension}
%    \begin{figure}
%        \centering
%        \includegraphicscopyright[width=0.9\linewidth]{nonlinear-analyses/cao}{
%	Figure is adapted from Cao L 1997 in {\bf Physica D}}
%	\caption{(A,B) $E_1(m)$ and (C, D) $E_2(m)$ values for (E) chaotic 
%		and (F) random time series} 
%   \end{figure}
%	
%\end{frame}
%}
%
%
%
%\subsection{Estimation of Embedding Parameters}
%%%%%%%%%%%%%%%%%%%%%%%%%%%%%%%%%%%%%%%%%%%%%%%%%%%%%%%%%
%{
%\paper{Cao L. 1997 in {\bf Physica D}}
%
%\begin{frame}{Estimation of Embedding Parameters }
%
%\begin{block}{False Nearest Neighbours (FNN)}
%\begin{equation*}
%E(m) &= \frac{1}{N-m\tau} \sum_{i=1}^{N-m\tau} 
%       \frac{ || \boldsymbol{X}_i(m+1) - \boldsymbol{X}_{n(i,m)}(m+1) || }
%            { || \boldsymbol{X}_i(m) - \boldsymbol{X}_{n(i,m)}(m) ||  } 
%\end{equation*}
%\end{block}
%
%\begin{block}{ $E_1(m)$ and $E_2(m)$ }
%\begin{equation*}
%E_1(m) = \frac{ E(m+1) } { E(m)} \quad 
%E_2(m) = \frac{ E^* (m+1) } { E^*(m)}
%\end{equation*}
%\end{block}
%
%
%
%
%	
%\end{frame}
%}
%
%
%
%
%\subsection{}
%%%%%%%%%%%%%%%%%%%%%%%%%%%%%%%%%%%%%%%%%%%%%%%%%%%%%%%%%
%{
%
%\begin{frame}{Average Mutual Information (AMI) for embedding delay}
%    \begin{figure}
%        \centering
%        \includegraphicscopyright[width=0.7\linewidth]{nonlinear-analyses/ami}
%	{Figure is adapted from Kabiraj et al. 2012 in {\bf Chaos}}
%	\caption{(A, B) AMI values for (C) chaotic and (D) noise time series.} 
%   \end{figure}
%	
%\end{frame}
%}
%
%
%
%
%
%\subsection{Estimation of Embedding Parameters}
%%%%%%%%%%%%%%%%%%%%%%%%%%%%%%%%%%%%%%%%%%%%%%%%%%%%%%%%%
%{
%\paper{Kabiraj et al. 2012 in {\bf Chaos}}
%
%\begin{frame}{Estimation of Embedding Parameters }
%
%\begin{block}{Average Mutual Information (ANN)}
%\begin{equation*}
%I(\tau) = \sum_{i,j}^N p_{ij} log_2 \frac{ p_{ij} }{ p_i p_j }.
%\end{equation*}
%\end{block}
%
%\end{frame}
%}
%
%
%
%
%
%\subsection{}
%%%%%%%%%%%%%%%%%%%%%%%%%%%%%%%%%%%%%%%%%%%%%%%%%%%%%%%%%
%{
%\paper{Eckmann et al. 1987 in {\bf Europhysics Letters}}
%
%\begin{frame}{Recurrence Plot}
%    \begin{figure}
%        \includegraphicscopyright[width=1.0\linewidth]{nonlinear-analyses/rp}
%		{Figure is adapted from (Marwan et al. 2007)}
%	\caption{(A) State space for Lorenz systems, and 
%		(B) Recurrence plot with embeddings ($m=1$, $\tau=1$) and $\epsilon=5$} 
%   \end{figure}
%
%\end{frame}
%}
%
%
%
%\subsection{RP and RQA}
%%%%%%%%%%%%%%%%%%%%%%%%%%%%%%%%%%%%%%%%%%%%%%%%%%%%%%%%%
%{
%\paper{Eckmann et al. 1987 in {\bf Europhysics Letters}}
%
%\begin{frame}{Recurrence Plots}
%
%
%$\mathbf{R}^{m}_{i,j} (\epsilon)$ is two dimensional plot of $N \times N$ square matrix
%defined by
%
%%%********************************[EQUATION]************************************
%\begin{equation*}
%\mathbf{R}^{m}_{i,j} (\epsilon) = 
%\Theta ( \epsilon_i - \Vert X(i) - X(j) \Vert ), 
%\quad i,j=1,\dots,N
%\end{equation*}
%%%********************************[EQUATION]************************************
%
%where $N$ is the number of considered reconstructed states of $X(i)$
%($X(i) \in \mathbb{R}^m$), 
%$\epsilon$ is a threshold distance, 
%$ \Vert  \cdot \Vert$ a norm, 
%and $\Theta( \cdot )$ is the Heaviside function.
%
%\end{frame}
%}
%
%
%
%
%\subsection{}
%%%%%%%%%%%%%%%%%%%%%%%%%%%%%%%%%%%%%%%%%%%%%%%%%%%%%%%%%
%{
%\paper{Marwan et al. 2007 in {\bf Physics Reports}}
%
%\begin{frame}{Recurrence Plot Patterns}
%    \begin{figure}
%        \includegraphicscopyright[width=\linewidth]{nonlinear-analyses/rpsp}{Figure is adapted from (Marwan et al. 2007)}
%	\caption{Recurrence plots for (A) uniformly distributed noise,
%		(B) super-positionet harmonic oscillation,
%		(C) drift logistic map with a linear increase term, and
%		(D) disrupted brownian motion.
%		} 
%   \end{figure}
%	
%\end{frame}
%}
%
%
%\subsection{}
%%%%%%%%%%%%%%%%%%%%%%%%%%%%%%%%%%%%%%%%%%%%%%%%%%%%%%%%%
%{
%\paper{Marwan and Webber, 2015}
%
%\begin{frame}{Recurrence Quantification Analysis (RQA)}
%
%\begin{description}
%\item [ \textbf{REC} ] enumerates the black dots in the RP.
%%%********************************[EQUATION]************************************
%\begin{equation*}
%	REC(\epsilon,N)= 
%	\frac{1}{N^2 - N} \sum^{N}_{i \neq j = 1} 
%	\mathbf{R}^{m}_{i,j}(\epsilon)
%\end{equation*}
%%%********************************[EQUATION]************************************
%\item [ \textbf{DET} ] fraction of recurrence points that form diagonal lines. \\
%			\textit{(interpreted as the predictability where, for example,
%				periodic signals show longer diagonal lines 
%				than chaotic ones.
%				)}
%%%********************************[EQUATION]************************************
%\begin{equation*}
%	DET=\frac{\sum^{N}_{l=d_{min}} l H_D{l} }{\sum^{N}_{i,j=1} 
%	\mathbf{R}^{m}_{i,j}(\epsilon) }
%\end{equation*}
%%%********************************[EQUATION]************************************
%
%\end{description}
%
%
%	
%\end{frame}
%}
%
%
%
%
%
%\subsection{}
%%%%%%%%%%%%%%%%%%%%%%%%%%%%%%%%%%%%%%%%%%%%%%%%%%%%%%%%%
%{
%%\paper{Marwan et al. 2007 in {\bf Physics Reports}}
%\paper{Marwan and Webber, 2015}
%
%\begin{frame}{Recurrence Quantification Analysis (RQA)}
%
%\begin{description}
%\item [ \textbf{RATIO} ] is the radio of DET to REC. \\
%			\textit{(useful to discover dynamic transitions)}.
%\item [ \textbf{ENTR} ] Shannon entropy of the frequency distribution of the 
%			diagonal line lengths.
%			\textit{(useful to represent the complexity of the 
%				structure of the time series)}
%%%********************************[EQUATION]************************************
%\begin{equation*}
%	ENT= - \sum^{N}_{l=d_{min}} p(l) ln p(l),
%\end{equation*}
%%%********************************[EQUATION]************************************
%where 
%%%********************************[EQUATION]************************************
%\begin{equation*}
%	p(l)=\frac{ H_D(l) }{ \sum^{N}_{ l=d_{min} } H_D(l) }
%\end{equation*}
%%%********************************[EQUATION]************************************
%
%\end{description}
%
%
%	
%\end{frame}
%}
%
%
%


