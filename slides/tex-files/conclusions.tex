\section{Conclusions}

\subsection{}
%%%%%%%%%%%%%%%%%%%%%%%%%%%%%%%%%%%%%%%%%%%%%%%%%%%%%%%%
{
\begin{frame}{Concluding Remarks}

\begin{itemize}
	\item What to quantify in movement variability? \\
	\textit{Complexity of movement based on the degrees of freedom of a person
	performing a certain task in a defined environment.}

	\item Which methods of nonlinear analysis are appropriate to quantify movement and
	how methods of nonlinear analysis are affected by real-world time series data ?\\
	\textit{Shannon entropy using 3D surface plots of RQA 
	appear to be robust to real-word data (i.e. different time series
	structures, window length size and levels of smoothness).}
%	\url{https://arxiv.org/abs/1810.09249}}

	\item What are these techniques good for?\\
	\textit{Quantification of skill learning in HRI, 
	dynamics of facial expressions, 
	fetal behavioral development, or
	nonlinear biomedical signal processing.}
\end{itemize}

%\badge{/badge/badge_v00}
\end{frame}
}

%\subsection{}
%%%%%%%%%%%%%%%%%%%%%%%%%%%%%%%%%%%%%%%%%%%%%%%%%%%%%%%%%
%{
%%\paper{Xochicale et al. 2018 in Progress}
%
%\begin{frame}{Future Work}
%
%Investigate:
%\begin{itemize}
%	\item other derivatives of acceleration data
%	to have better understanding of the nature of human movement,
%	\item other methodologies for state space reconstruction,
%	\item the robustness of Entropy measurements with RQA, and 
%	\item variability in perception of velocity.
%\end{itemize}
%
%Apply the proposed method in the context of human-humanoid interaction to:
%\begin{itemize}
%	\item evaluate improvement of movement performance,
%	\item provide feedback of level of skillfulness, and 
%	\item quantify motor control problems and pathologies.
%\end{itemize}
%
%
%\end{frame}
%}
%


